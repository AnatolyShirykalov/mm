\section{Лекция 23.10}
В прошлый раз мы получили уравнения гибкой палки, рассматривали бегущую волну. Сегодня у нас будет стоячая волна. Эта задача обычно предполагает, что у нас есть некоторые граничные условия и в этом случае мы можем найти решение в виде стоячей волны методом разделения переменных.

Итак у нас есть уравнение Эйлера гибкой палки (см рис. \ref{fig:beam})
\[
  \CP{^2u}{t^2} + c^2 \CP{^4u}{x^4} = 0,
\]
где $c^2 = \frac{EI}{S\rho} = \frac1{12} \frac E\rho h^2$, $I = \frac{b h^3}{12}$

Параметр $b$ не входит в уравнение вообще. Вы можете представить две палки рядом, они будут осцилировать как одна, так как $b$ направлено перпендикулярно доске, а все колебания происходят в плоскости доски.

В прошлый раз мы обсуждали бегущую волну. Теперь мы обсуждаем стоячую. Предположим, у нас есть некоторые краевые условия, а именно
\begin{figure}[H]
  \centering
  \tpximg{boundary}
  \caption{boundary}
  \label{fig:boundary}
\end{figure}
Наши граничные условия:
\[
  u(0,t) = 0;\quad
  u'(0,t) = 0;\quad
  u''(l,t) = 0;\quad
  u'''(l,t) = 0.
\]
Слева жёсткая заделка, там всё понятно. На правом конце интереснее. Мы считаем, что нет никаких внешних моментов, отсюда и третье условие (кривизна на свободном конце равна нулю, иначе было бы некое напряжение, которое не понятно чем компенсировать). На счёт четвёртого условия как-то мутно сказано (продифференцируем мол второю производную и получим).

Четвёртого порядка уравнение с четырьмя граничными условиями.

Будем искать решение в виде произведения
\[
  u = \phi(x)\theta(t);
\]

Подставляя такую замену, получим
\[
  \phi\theta'' + c^2 \tau \phi^{(4)} = 0.
\]

Как обычно, поделим это уравнение на произведение $\phi\theta$. Получим, окончатльно
\[
  \frac1\theta \theta'' = - c^2 \frac1\phi \phi^{(4)} = -\omega^2.
\]

Абсолютно такие же действия мы делали для колебаний струны. Получаем два обыкновенных дифференциальных уравнения, так как два выражения, зависящие от разных переменных, оказываются равны, а значит, они равны некоторой константе. Из физических соображений констана отрицательная (без физических соображения при попытке использовать неотрицательную константу можно получить только решения, которые не могут удовлетворить краевым условиям).
Итак, система уравнения
\[
  \begin{cases}
    \theta'' = \omega^2 \theta = 0;\\
    \phi^{(4)} - k^2 \phi = 0,& k^2 = \frac{\omega^2}{c^2}.
  \end{cases}
\]

Для уравнения на $\phi$ у нас есть начальные и конечные условия. Можем найти характеристическое уравнение, его корни.
\[
  \lambda^4 - k^2  = 0;
  \quad \imp \quad
  \lambda^2 = \pm k.
\]

Мы можем обозначить $k=\alpha^2$. Тогда четыре корня имеют вид
\[
  \lambda_1, \lambda_2 = \pm \alpha;\quad
  \lambda_3, \lambda_4 = \pm i\alpha.
\]

На графике эти корни выглядят следующим образом
\begin{figure}[H]
  \centering
  \tpximg{roots}
  \caption{roots}
  \label{fig:roots}
\end{figure}

Можем написать общее решение в комплексной форме
\[
  \phi = a_1 e^{\alpha x} + a_2 e^{-\alpha x} + a_3 e^{i\alpha x} + a_4 e^{-i\alpha x}.
\]

Но мы также можем найти вещественные решения
\[
  \phi = c_1 \sh \alpha x + c_2 \ch \alpha x + c_3 \sin \alpha x + c_4 \cos \alpha x.
\]

Начинаем подставлять начальные и конечные условия. Из условия $\phi(0) = 0$ вытекает, что
\[
  c_2 + c_4 = 0;
\]
Из того, что $\phi'(0)$, мы получим
\[
  c_1 + c_3 = 0.
\]

Исходя из $\phi''(l) = 0$ мы получаем (при дифференцировании каждого слагаемого вылезет множитель $\pm\alpha^2$, мы его можем сразу сократить, но знаки не забудем).
\[
  c_1 (\sh \alpha l  + \sin \alpha l) + c_2 (\ch\alpha l + \cos \alpha l) = 0.
\]
Из последнего условия получаем
\[
  c_1 (\ch \alpha l + \cos \alpha l) + c_2 (\sh \alpha l - \sin \alpha l) = 0.
\]

Имеем два уравнения на две неизвестных. Каково условие существования нетривиального решения: нулевой определитель системы
\begin{multline*}
  \begin{vmatrix}
    \sh\alpha l + \sin \alpha l & \ch \alpha l + \cos \alpha l \\
    \ch \alpha l + \cos \alpha l& \sh\alpha l - \sin \alpha l
  \end{vmatrix} = \\
   = \sh^2\alpha l - \sin^2 \alpha l - \ch^2 \alpha l - 2\ch \alpha l \cos \alpha l - \cos^2\alpha l = -2 - 2\ch\alpha l \cos \alpha l.
 \end{multline*}

 Что нам это даёт. Мы можем показать это графически. Я буду рисовать произведение, которое должно быть равным $-1$.
 \begin{figure}[H]
   \centering
   \tpximg{al}
   \caption{alphal}
   \label{fig:alphal}
 \end{figure}
 Итак нас интересует некое семейство $\alpha_n l \to \frac\pi2 n$ при $n\to\infty$. Примеры значений $\alpha_1 l = 1.85$, $\alpha_2 l = 4.69$, $\alpha_3 l = 7.85$.

 Картинки примерно такие
 \begin{figure}[H]
   \centering
   \tpximg{3modes}
   \caption{modes}
   \label{fig:modes}
 \end{figure}

 На практике каринка как правило первая. Но если систему сделать хорошо и соответствующим образом возбудив можно получить не первую моду.
