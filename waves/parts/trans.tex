\newcommand{\eexp}{\exp\big[i(\omega t - \boldsymbol\kappa\cdot \boldsymbol r)\big]}
\section{Общая теория волн воды, порождённых осциллирующими источниками}
Как уе было отмечено в предисловии, подбор начальных условий в общей постановке задачи о волнах задача, требующая математической аккуратности и понимания физической реальности. Итак мы рассматриваем проблему начальных условий для двумерной волны воды, порождённой источником фиксированной частоты. Для начала мы рассмотрим решение вида плоской фолны, которое можно записать в виде формулы:
\[
  q = a \eexp,
\]
где $a, \omega, \kappa = (k, l)$ "--- заданы и $r=(x,y)$. Мы полагаем, что конкретное граничное значение $\eta$ на некоторой граничной плоскости $z=\const$ может быть представлено в виде
\[
  \eta = a D(\omega, \kappa)\eexp.
\]

Если 
