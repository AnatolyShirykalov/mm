\section{30.10}
Сегодня мы рассмотрим вынужденные колебания балки. В прошлом мы рассматривали свободные колебания, то есть без внешних сил.
Сегодня на систему будет действовать периодическая сила.

Как и в дискретных системах, у нас будет несколько типов колебаний. Для математического маятника у нас есть свободные, вынужденные, параметрические и авто-колебания.

Теперь вопрос: кто может обхяснить по-русски.
Свободные: линейку за хвостик и отпустили.
Вынужденные: прилагаем периодическую силу или момент к маятнику.
Параметрические: когда меняется в системе какой-то параметр. Для маятника это точка подвеса перемещается периодически вертикально.
Автоколебания: на систему не действуют периодические силы. Но система самовозбуждается. Это например, грузик на ленте транспортёра. Грузик с пружинкой. То транспортёр грузик увлекает, топружина тащит обратно.

В распределённых системах примеров ещё больше. Стоит лес, дует ветер, деревья качаются. Это не периодическая сила, это автоколебания. Второй пример скрипка: гладкий смычок по гладкой струне.


\subsection{Дискретная система}
Рассмотрим маятник
\[
  \ddot z + z  = \mu \sin \omega t;
\]
Если $\mu = 0$, у нас свободные колебания. Иначе вынужденные.

Ищем решение в виде
\[
  z = A \sin(\omega t + \alpha );
\]

Подставляем его в уравнение
\[
  (1 - \omega^2) A \sin (\omega t + \alpha) = \mu\sin\omega t.
\]

У нас тут некая фаза $\alpha$, не знаем какая. Будем пытаться оценить. У нас здесь две независимых функции $\sin(\omega t)$ и $\cos(\omega t)$. Можем написать уравнения на коэффициенты при этих функциях.
\begin{eqnarray*}
  \sin(\omega t)\colon && (1-\omega^2)A\sin\alpha = 0;\\
  \cos(\omega t)\colon && (1-\omega^2)A\cos\alpha = \mu.
\end{eqnarray*}

Видим, что если $\sin\alpha\ne 0$, то $\omega^2 = 1$. Это означает, что второе уравнение не удовлетворяется, что нехорошо.

Стало быть у нас $\sin\alpha = 0$, а значит, $\cos\alpha = \pm 1$. Следовательно, $A = \pm \frac{\mu}{1-\omega^2}$.
Движение происходит либо в фазе, либо в противофазе.

У нас есть особая точка $\omega = 1$, где амплитуда бесконечна, эта ситуация называется резонансом.

Зависимость амплитуды от частоты имеет следующий вид
\begin{figure}[H]
  \centering
  \tpximg{omegaa}
  \caption{$A(\omega)$}
  \label{fig:omegaa}
\end{figure}

Две ветки. Слева движение в противофазе, справа --- в фазе. Есть точка, где амплитуда бесконечна. Если добавить вязкое трение, амплитуда станет ограниченной (это вторая линия на рисунке).

\subsection{Распределённая система: балка}
\begin{figure}[H]
  \centering
  \tpximg{forcedbeam}
  \caption{forced Beam}
  \label{fig:forcedbeam}
\end{figure}
У нас есть две системы координат. Одна $(x,u)$ неподвижна, $(y, u) $ подвижная, связанная с точкой крепления балки.

Как правило такие задачи проще решать в подвижной системе координат. Для состовления уравнений движения придётся вводить силы инерции.
Пара слов о силах инерции. Периодически возникали споры: существуют они или нет. Одна из таких дискуссий была между академиками Мышлинским и Седовым. Это была борьба местами выходящая за рамки принятых правил.
Моя точка зрения: в данном случае это чисто математический приём, силы инерции являются результатом наших математических преобразованием.


Если мы рассмотрим некоторый кусочек длины $\Delta$ нашей балки, можем написать второй  закон для системы без внешней силы.
\[
  \Delta S \rho \CP{^2u}{t^2} + \Delta E I\CP{^4u}{x^4} = 0;
\]

Если мы переходим в подвижную систему, нужно справа добавить $- m \ddot\delta$.
\[
  \Delta S \rho \CP{^2u}{t^2} + \Delta E I \CP{^4u}{y^4} = (\omega^2 a\sin\omega t)\Delta S \rho.
\]

Далее мы приводим уравнение к виду
\[
  \CP{^2u}{t^2} + c^2 \CP{^4u}{y^4} = \mu \sin\omega t,
  \quad \mu = a\omega^2.
\]

Дальше везде дальше будем писать $x$ вместо $y$.

Мы видим, что уравнение выглядит похоже на дискретное. Мы можем найти некоторое похожее решение. Будем искать решение в виде
\[
  u = \phi(x) \cdot \sin (\omega t + \alpha).
\]

Подставляем в уравнение
\[
  \big[- \omega^2 \phi + c^2 \phi''''\big]\big(\sin\omega t \cos\alpha + \cos\omega t\cos\alpha\big) = \mu \sin\omega t.
\]

Приравняем коээфициенты при незавимых по функций по аргументу $t$
\begin{eqnarray*}
  \cos\omega t\colon&&\big(c^2 \phi'''' - \omega^2\phi\big)\sin\alpha = 0;\\
  \sin\omega t\colon&&\big(c^2 \phi'''' - \omega^2\phi\big)\cos\alpha = \mu.
\end{eqnarray*}

Получили уравнения, похожие на те, что мы получили в дискретном случае.

Снова работает рассуждение: если $\sin \alpha \ne 0$, то второе уравнение не выполняется. Значит, $\sin \alpha = 0$ и $\cos\alpha = \pm 1$.

Давайте не будем писать $\pm$, зато будем подразумевать, что $\mu$ может иметь разные знаки.

Осталось решить уравнение
\[
  c^2 \phi'''' - \omega^2 \phi = \mu.
\]
Решением будет сумма общего однородного и частного неоднородного. Не будем писать характеристическое уравнения, находить корни, мы это уже делали в прошлый раз.
\[
  \phi = \phi_h + \phi_p.
\]
Обозначения $h$ --- homogenous, $p$ --- partial.

Легко найти частное решение
\[
  \phi_p = -\frac\mu{\omega^2} = -a.
\]

Общее решение с прошлого раза
\[
  \phi_h = c_1 \sh \alpha x + c_2 \ch \alpha x + c_3 \sin \alpha x + c_4 \cos \alpha x,
  \quad \alpha = \sqrt{\frac{\omega}{c}}.
\]

Итого
\[
  \phi = c_1 \sh \alpha x + c_2 \ch \alpha x + c_3 \sin \alpha x + c_4 \cos \alpha x - a.
\]

Поставим задачу на граничные условия. Слева конец жёстко закреплён
\[
  \phi(0) = \phi'(0) = 0,
\]
справа конец свободен, на него не действуют моменты
\[
  \phi''(l) = \phi'''(l) = 0.
\]

Если в прошлый раз, мы могли сразу получить простые соотношения вида $c_1 = -c_3$, потому что у нас есть ненулевое частное решение. В данном случае перед нами полная система линейных уравнений четвёртого порядка со следующей матрицей (здесь вместо $\sin\alpha l$ будем писать $\sin$, аналогично остальных тригонометрических функций).
\[
  \begin{pmatrix}
    0 & 1 & 0 & 1\\
    1 & 0 & 1 & 0\\
    \sh & \ch & -\sin & -\cos\\
    \ch & \sh & -\cos & \sin
  \end{pmatrix} \begin{pmatrix}
    c_1 \\ c_2 \\ c_3 \\ c_4
  \end{pmatrix} = \alpha\begin{pmatrix}
    1 \\ 0 \\ 0 \\ 0
  \end{pmatrix}.
\]

Если взять определитель этой матрицы, то
\[
  \det  = 2 (1 + \ch \alpha l \cos\alpha l).
\]

В прошлый раз ненулевое решение существовало если ровно такая же конструкция была равна нулю.

При нулевом определителе у нас некоторые $c_j$ могут принять бесконечное значение. Получаем похожую картинку.
\begin{figure}[H]
  \centering
  \tpximg{forcedomega}
  \caption{forced omegas}
  \label{fig:forcedomega}
\end{figure}
Здесь $\omega_k = c^2 \alpha^2_k$, где $\alpha_k\approx (2k - 1)\frac\pi2$.

Снова если добавить вязкое трение, можно получить вторую линию амлитуд. Любая реальная система имеет некоторую диссипацию, и реальная картинка похожа на нижнюю линию.
