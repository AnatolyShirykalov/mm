\section{13.11}
Сегодня параметрический резонанс. Возникает как в конечномерных, так и в распределённых системах.

Типичный представитель в дискретных системах: уравнение Матье.
\[
  \ddot x + \omega^2(1+\e\sin 2t)x = 0;
\]

Если $\e = 0$, то это уравнение превращается в уравнение гармонического осциллятора. Если исследовать уравнение Матье, можно получить, что при $\omega \in \N$, то уравнение имеет резонанс.
\begin{figure}[H]
  \centering
  \tpximg{momatie}
  \caption{Matie}
  \label{fig:matie}
\end{figure}

Перейдём к рассмотрению параметрического резонанса в струне. Мы знаем, что частота струны зависит от силы натяжения на концах $T$.
\begin{figure}[H]
  \centering
  \tpximg{parametrical}
  \caption{parametrical}
  \label{figja:<+label+>}
\end{figure}

Переда нами обычная струна, то есть её уравнение это обычное уравнение струны
\[
  \CP{^2u}{t^2} - c^2 \CP{^2}{x^2} = 0,\quad c^2 = \frac T\rho.
\]

Предполагаем, что $T = T_0(1 + \gamma \cos\theta t)$.

Будем искать решение в виде $u = \phi(x)  q(t)$. Подставляя в уравнение, получаем
\[
  \frac1{c^2}\frac1q\DP{^2q}{t^2} = \frac1\phi\DP{^2\phi}{x^2} = -k^2.
\]
Пишем на каждую неизвестную независимое уравнение
\[
  \DP{^2\phi}{x^2} + k^2\phi = 0;\qquad
  \DP{^2q}{t^2} + c^2 k^2q = 0.
\]

Итак, два уравнения выглядят, как осцилляторы, но во втором уравнении мы имеем параметр $c$, который зависит от $T$. Уравнение аналогично уравнению Матье.

Первое уравнение вообще уравнение осциллятора и не зависит от времени. Тут у нас есть граничные условия, а именно $\phi(0) = \phi(l) = 0$. Чтобы этому удовлетворить,
\[
  \phi = A_n \sin k_n x, \quad k_n = \frac{n\pi}l.
\]

Второе уравнение переписываем как
\[
  \DP{^2q}{t^2} + \Omega^2_n (1 + \e \cos\theta t)q = 0.
\]

Здесь $\Omega^2_n = c^2 k^2_n= \frac T\rho k^2_n$.

Для каждого $n$ у нас есть уравнение Матье.
Напишем результат, аналогичный результату исследования уравнени Матье.
\[
  \frac{\Omega_n}{\theta_j} = \frac j2
\]
является резонансным случаем (уравнение на нахождение резонансной частоты~$\theta_j$).

Это источник многих неприятностей в технике. Дёргаем струну, а у нас возникают поперечные колебания, да ещё и с другими частотами.

\subsection{Как связаны разные виды колебаний}
Пусть есть уравнение
\[
  \CP{^2u}{t^2} - c^2\CP{^2u}{x^2} = 0.
\]

Если мы ищем решение в виде $u = \phi(kx - \omega t) = \phi(\xi)$, то подставив, получим
\[
  \omega^2 \phi'' - c^2k^2\phi'' = 0.
\]

Такое возможно, если $\omega = \pm ck$. Тогда
\[
  \phi = A \sin(x - ct) + A \sin(x + ct).
\]
В результате тригонометрических преобразований можно получить
\[
  \phi = 2A\sin x\cos ct.
\]
Эта форма совпала с тем же видом решения, которое мы бы получили методом разделения переменных.

В данном случае метод поиска стоячей волны годится для поиска бегущей волны.
Стоячая волна --- это две бегущих.
