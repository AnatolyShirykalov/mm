\section{Лекция от 16.10}

We know that something shows the motion of the string;
В прошлый раз мы выяснили, что волновое уравнение описывает поперечное колебание струны и продольное движение твёрдого тела, фактически распространение звуковых колебаний.

В прошлый раз рассматривали движение бегущей волны. Сегодня посмотрим на стоячую волну.

We will start with method of variable separation.
\[
  \CP{^2u}{t^2} - c^2 \CP{^2u}{x^2} = 0;
\]

Let's consider the next task.
\begin{figure}[H]
  \centering
  \tpximg{1}
  \caption{Волна}
  \label{fig:wave}
\end{figure}

Let's make some tricks
\begin{eqnarray*}
  u &=&  Y(x) \cdot \theta(t);\\
  Y\cdot \theta'' - c^2 \theta Y'' &=&  0\big| \frac1{Y\theta};\\
  \frac1\theta \theta'' = c^2 \frac 1Y'' &=&  -\omega^2.
\end{eqnarray*}
Итак, это раздеение переменный. Поделили подставили в волновое уравнение в виде произведения функций от одной переменной. Знак $\omega^2$ выбирается из физических соображение, с ним получается реализуемое физическое решение.

Получаются два уравнения
\begin{eqnarray*}
  \theta'' + \omega^2\theta &=&  0;\\
  Y'' + \frac{\omega^2}{c^2} Y &=& 0.
\end{eqnarray*}

Теперь мы имеем два обыкновенных дифференциальных уравнениея.

Let begin with the second equation. We will find the solution such as (выбор решение таков, чтобы сразу первое граничное условие выполнлось)
\[
  Y = A \sin kx,\pau k = \frac\omega c.
\]
The solution has to satisfy the boundary condition. To satisfy the right boundary condition with should have a numbers $\lambda = \frac{2\pi}k$, so this condition:
\[
  u(l) = 0 \imp l = n\frac\pi k = n\frac\lambda2.
\]

Можем написать $k_n = \frac{n\pi}l$.

Мы можем заметить также, что благодаря соотношению $k=\frac\omega c$, мы имеем, что
\[
  \omega_n = c k_n = n\frac{c\pi}l.
\]

Таким образом, мы получаем некий дискретный спектр колебаний, который возможен при наших начальных условиях. Наличие граничных условий приводит к тому, что в решение появляется дискретный спектр. Итого решение будет представлять собой комбинацию решений найденного типа с $k $ из этого набора. Финальное решение
\[
  u =  \sum A_i \sin(k_i x)\sin(\omega_i t).
\]

Если у нас будет только одна гармоника, то мы как раз увидим стоячую волну. Примерно такая картинка будет, как на гитаре
\begin{figure}[H]
  \centering
  \tpximg{2}
  \caption{Стоячая волна, одна гармоника}
  \label{fig:fixedwave}
\end{figure}

Вторую гармонику на гитаре вызвать сложно.
Не всегда по явлению видно, как это записать математически.

\subsection{Пример поперечны волн в некоем трёрдом стержне}
Let us consider a beam
\begin{figure}[H]
  \centering
  \tpximg{beam}
  \caption{Beam}
  \label{fig:beam}
\end{figure}

Let us consider a piece.
\begin{figure}[H]
  \centering
  \tpximg{piece}
  \caption{Piece}
  \label{fig:piece}
\end{figure}

Считаем, что плоские кусочки остаются плоскими и деформация выглядит как сектор радиуса $R$.
\[
  dF(\xi) = \frac{\Delta l}l b E d\xi.
\]
Здесь $E$ "--- модуль Юнга.

Чем дальше от средней линии, силы будут возрастать
\[
  \frac{\Delta l}{l} = \frac{\xi\alpha}{l_0}.
\]
Очевидно, что зависимость длины от $\xi$
\[
  l(\xi) = (R + \xi)\alpha = \frac{\xi}R.
\]

И тогда
\[
  F(\xi) = \frac{ Eb}{R} \xi d\xi
\]

Чтобы получить момент, нужно проинтегрировать эту силу
\[
  M = \int\limits_{-h/2}^{h/2} \xi \cdot dF(\xi) = \frac{Eb}{R}\int\limits_{-h/2}^{h/2} \xi^2\,d\xi = \frac23 \frac{Eb}{R}\left(\frac h2\right)^3 = \frac1{12} \frac{Eb}Rh^3.
\]

Есть некоторое специальное обозначение для момента инерции относительно средней линии
\[
  I = \frac1{12} bh^3.
\]

Мы связали момент с неким радиусом кривизны $R$.
Если мы рассматриваем некий участок кривой линии.
\begin{figure}[H]
  \centering
  \tpximg{theta}
  \caption{Theta}
  \label{fig:theta}
\end{figure}
Имеем,
\[
   \alpha = \theta(x) - \theta(x+ dx) = -\CP{^2u}{x^2}\,dx.
\]
При этом
\[
  dx = R\alpha = - R\CP{^2u}{x^2}\,dx.
\]
А стало быть мы получаем, что
\[
  R = \frac 1{u_{xx}}.
\]

Итак получаем, что момент
\[
  M = - IE u_{xx}.
\]

Но и этого недостаточно, чтобы написать уравнение движения.

Мы считаем, что для малых колебаний интересны только перемещения вдоль $u$. А какие вертикальные силы у нас действуют.
\begin{figure}[H]
  \centering
  \tpximg{mom}
  \caption{Mom}
  \label{fig:mom}
\end{figure}

Нужно брать разные знаки в $x$ и в $x + dx$, записывая правую часть.
\[
  Q (x + dx) \cdot \frac{dx}2 + Q(x) \cdot \frac{dx}2 - M(x + dx) + M(x) = 0.
\]

Можем переписать эти уравнения, учитывая только первое приближение по $dx$.
\[
  Q\,dx = \CP Mx\,dx;\quad Q = \CP Mx
\]

На самом деле эти уравнения не очень точный. Может ещё быть ненулевой момент инерции. Но так как кусочек маленький, движение очень маленькое, мы пренебрегаем моментом инерции.

Теперь мы имеем выражение для $Q$. Можем написать закон Ньютона для нашего кусочка в проекции на ось $u$. Объём кусочка $bh\,dx$.
\[
  \rho\cdot b h \,dx\CP{^2u}{t^2} = Q(x+dx) - Q(x) = \CP{Q}{x}\,dx.
\]

Далее мы заменяем момент полученным недавно выражением
\[
  \rho bh \CP{^2u}{t^2} = \CP{^2M}{x^2}.
\]

Теперь у нас есть выражение для собственно момента $M$ через $u$. Наконец, имеем
\[
  \rho b h \CP{^2u}{t^2} = -IE\CP{^4u}{x^4}
\]

И совсем уже наконец, мы имеем уравнение следующего вида
\[
  \CP{^2u}{t^2} + \frac{IE}{\rho b h} \CP{^4u}{x^4} = 0.
\]

Это уравнения описывает движение гибкой палки.

В результате пришли к уравнению Эйлера для твёрдого стержня. Оно похоже немножко на волновое уравнение, но на самом деле принципиально отличается.
Простая достаточно кухня, но требует аккуратности. Она нам нужна для того, чтобы последовательно записать уравнение этого маленького кусочка под действием поперечных сил.
Сначала выразили как момент выражется через радиус кривизны (соответственно через вторую производную поперечного сечения). Затем из уравнения равновесия этого кусочка нашли как зависит сила в сечении $Q$ от перемещения. И дальше записали второй закон Ньютона уже имея, что на этот кусочек действуют две силы именно в этом поперечном направлении.
Единственное, что тут требует основания, это уравнение равновесия. ПО идее вместо него должен был быть закон Ньютона, но мы считаем, что момент инерции этого кусочка мал и им пренебрегаем, остаётся просто равновесие под действием сил. У нас тут $dx$ стремится у нулю, но и высота $h$ тоже должна быть невелика. Это натяжка в нашем уравнение, но зато получается красивое уравнение.

\subsection{Для чего}
Теперь мы посмотрим, что же нам даёт вот это уравнение, какие здесь возникают решения. Прежде всего будем искать решение в виде бегущей волны.

Ищем $u$ как
\[
  u = A \sin(kx - \omega t).
\]
Обычно вводят ещё переменную $\xi = kx - \omega t$.
Всё хозяйство обозначим через $c^2$
\[
  c^2 = \frac{IE}{\rho b h}.
\]

Если такое решение подставить просто на прямую в наше уравнение, при дифференцировании по $t$ даст нам дважды $\omega$ и $-$. В результате первый член даст нам
\[
  - A \omega^2 \sin(\xi) + c^2 k^4 A \sin(\xi) = 0;
\]

Соответственно (видно, что есть общий множитель), получаем соотношение
\[
  \omega^2 = c^2k^4\qquad \imp \qquad \omega = ck^2.
\]

Если мы зафиксируем $\xi=\const$ (завиксировали фазу, как обычно, коггда работаем с бегущими волнами), фазовая скорость имеем $v_\text{ф} = \frac{\omega}{k} = ck$.

Напоминаю, что чем больше $k$, тем меньше длина волны. Волновое число.

Можем записать $v_\text{ф} = c\frac{2\pi}\lambda$. Чем больше длина волны, тем меньше фазовая скорость.
Для волнового уравнения скорость была постоянна для любых частот.
Когда скорость зависит от длины волны, это называется дисперсия. Говорят, что в нашем случае дисперсия отрицательная, потому что скорость фазовая падает с увеличением длины волны.
Пример положительной дисперсии: волны на поверхности океана. Более длинные волны приходят быстрее на берег, потом подтягиваются меньшие.

В стержне мелкие волны распространяются быстрее, чем длинные. Когда приближается поезд, то сначала вы слышите высокочастотные звуки, можете обратить внимание вблизи железнодорожного переезда.

И ещё один момент, который надо с вами обсудить сегодня. Это скорость распространения волны.
Вот есть фазовая скорость. Возьмём реальный объект, например, металлическая линейка
\begin{figure}[H]
  \centering
  \tpximg{line}
  \caption{line}
  \label{fig:line}
\end{figure}
Параметры линейки $h = 0.5\text{мм} = 5\cdot 10^{-4}\text{м}$. При том $E = 2\cdot 10^{11}\text{Па}$, $b = 20\text{мм} = 2\cdot 10^{-2}\text{м}$, $\rho = 8\cdot 10^2 \text{кг}/\text{м}^3$.
Получаем $I = \frac{bh^3}{12}$, $c^2 = \frac{E h^2}{12\rho}$. В итоге
\[
  c^2 = \frac{2\cdot 10^{11}\cdot 25 \cdot 10^{-8}}{12\cdot 8\cdot 10^3} = \frac{50}{96} \approx 0.5.
\]

Получается, что $v_\text{ф} \sim 1\frac{\text{м}}{\text{с}}$.

Сравним с продольными волнами, которые мы получали ранее. Там у нас
\[
  c^2 = \frac E\rho= \frac{2\cdot 10^{11}\text{м}^2}{8\cdot 10^3\text{с}^2} = 2.5\cdot 10^7\frac{\text{м}^2}{\text{с}^2}.
\]

В итоге $c = 5\cdot 10^3\frac{\text{м}}{\text{с}}$. Такая скорость распространения звука в металле. Если ударить рельсу кувалдой, пойдёт две волны. Звуковая волна пойдёт быстро, а деформация пойдёт гораздо медленнее.
